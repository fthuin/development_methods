\documentclass[a4paper, 11pt]{article}

\usepackage[french]{babel}
\usepackage[utf8]{inputenc}
\usepackage[T1]{fontenc}
\usepackage{placeins}
\usepackage{csquotes}
\usepackage{hyperref}
\usepackage{tikz}
\usetikzlibrary{automata,arrows,positioning,shapes}
\usepackage[left=2cm,right=2cm,top=2cm,bottom=2cm]{geometry}

\author{Florian Thuin}
\date{\today}
\title{Presentation notes}

\begin{document}
    \maketitle
    \tableofcontents

    2014 graduate

    \section{About Actito}

    \subsection{The company}

    Louvain-la-Neuve. Old Citobi. 50+ people. Created 2000. RTL-TVI, Kia, Lunch Garden,
    L'Oréal, PartenaMut, ABB, Generali, La grande récré, noukie's, toyota, SNCB, VW, Touring
    Ranstad, TEC, AXA

    France 2014

    \subsection{The software \textemdash{} Commercial point of view}

    Actito is a marketing automation software platform. Customization for each
    customer in large database of the client.

    Interact with customers,regardless of the channel

    Centralise all your data and interactions in one place

    The right content to the right person at the right time of their journey

    100\% in the Cloud

    \subsection{The software \textemdash{} Technical point of view}

    Our production servers are located somewhere in Belgium.

    The servers for our testing environment are stored at Actito headquarters in
    Louvain-la-Neuve

    Our infrastructure is in constant expansion.

    \subsubsection{From a monolithic to a microservices architecture}

    Not so far ago, Actito had a monolithic architecture with two big applications:
    Actito Core and Actito Web.

    Question: Can you list the issues that raise from a monolithic architecture
    when the system grows with many new functionalities?

    \begin{itemize}
        \item Strong coupling
        \item Conflict between technologies evolution (obligé de modifier certaines parties du code
        à cause de ça alors qu'on voulait pas y toucher)
        \item One change in the program oblige le déploiement de toute l'application.
        \item If you want to scale a part of your system, it is not possible you are forced to replicate all the application.
    \end{itemize}

    \begin{itemize}
        \item Deployment: interruption of service
        \item Hard to use new technologies
        \item Hard to maintain
        \item How to scale only a part of the system?
    \end{itemize}

    \subsubsection{We adopted an architecture with small single purpose applications}

    \begin{itemize}
        \item Very short interruption of service on one application at a time (not even
        true if stateless and replicated)
        \item Free to use the best new technologies
        \item We need to perform a maintenance on an old application only when it is
        absolutely necessary
        \item Some applications can be load balanced.
    \end{itemize}

    we have at least 40 applications but 20 we didn't touch in the last year.

    \subsubsection{We use mostly well known open source technologies}

    Do you know those technologies?

    Git, Spring, Bower, MAven, RabbitMQ, Tomcat, NGINC, Ember, MySQL, Cassandra, Hibernate

    \subsection{The IT team}

    \subsubsection{R\& D}

    \begin{itemize}
        \item Standard
        \item Scrum
        \item Product Owner
        \item 1 manager, 7 developers
        \item One manager
    \end{itemize}

    \subsubsection{Operations}

    \begin{itemize}
        \item Client specific
        \item Kanban
        \item Clients
        \item 4 Technical project managers, 4 developers
        \item Two managers
    \end{itemize}

    There is no good solution in software engineering. We have to do with the
    deadline of the client, with the kind of team we're dealing with, different
    kind of workers.

    \section{The development process}

    R\& D team with 8 engineers

    What to do?

    How to prioritize the work?

    How to split the work?

    How to monitor the progress and overcome
    % ???

    \subsection{Our Scrum methodology}

    Our work schedule is divided in Sprints of two weeks.

    A sprint contains stories.

    A story is a functional unit to be developed

    A story as a complexity, represented by points.

    We an compute the velocity of the team. It allows the Scrum master to
    know the quantity of stories that can be included in a single Sprint.

    The product owner is responsible of defining the stories and verifying that the developed
    story meets the specification.

    Pecheur: You have product owner, you have a manager/Scrum master. What is the difference
    between product owner and scrum manager?

    \subsubsection{High level functionalities of Actito are defined in scoping meeting}

    Meeting every 2 months between the direction, the Scrum master and the product owner.

    The philosophy of the software is discussed.

    High level functionalities are defined and prioritized.

    Those functionalities are too coarse-grained to be stories as-is.

    \begin{tikzpicture}[auto, on grid,node distance=3cm, align=center]
        \node[draw, text width=1.5cm] (SM) {Scoping meeting};
        \node[draw, right of=SM, text width=1.5cm] (PO) {Product owner}; % plays the role of a customer
        \node[draw, right of=PO, text width=1.5cm] (LS) {List of stories};
        \node[draw, right of=LS, text width=1.5cm] (Scrum) {Scrum meeting}; % team
        \node[draw, right of=Scrum, text width=1.75cm] (LSPE) {List of stories prioritized evaluates};
        \node[draw, right of=LSPE, text width=1.5cm] (ScrumMaster) {Scrum Master};
        \node[draw, below of=LS, text width=1.85cm] (TeamPP) {Team spokesman};
        \node[draw, below of=ScrumMaster, text width=1.5cm] (SC) {Sprint content};

        \path[->] (SM) edge node {} (PO);
        \path[->] (PO) edge node {} (LS);
        \path[->] (LS) edge node {} (Scrum);
        \path[->] (Scrum) edge node {} (LSPE);
        \path[->] (LSPE) edge node {} (ScrumMaster);
        \path[->] (TeamPP) edge node {} (LS);
        \path[->] (SC) edge node {} (ScrumMaster);
    \end{tikzpicture}

                                                                                                                                                                                                % Velocity        Availability(vacances)
                                                                                                                                                                                                %       \         /
    % Scoping meeting ----------------> Product owner (plays the role of a customer) --------> List of stories -----------------> Scrum meeting --------> List stories prioritized evaluates -----------> Scrum Master
    %                 high level functionalities                                                      ^                               (team)                                                                  |
    %                                                                                                 |                                                                                                       v
    %                                                                                            Scrum master                                                                                             Sprint content
    %                                                                                            (Team porte-parole)
    Specification are defined in JIRA.\@

    \subsubsection{Stories are defined by the Scrum master and the product owner.}

    Before the beginning of a Sprint, the Scrum master and the product owner meet to define
    the next stories and their priority.

    The stories are encoded in Jira.

    Other kind of tasks must be considered (Technical Analysis, Technical Task, Epic,\ldots)

    Example of a story: As a Actito User, I want to see my campaign transactions only for
    30 days.

    \subsubsection{Let's play Poker!}

    At the start of the Sprint, meeting with the whole team.

    The product owner explains the stories.

    The developers discuss and evaluate the complexity by playing planning poker. (Testing is included in the complexity)

    Very different estimations show that there is a misunderstanding of the story.

    \subsubsection{The Sprint can start}

    The Scrum master decides what stories will be included in the Sprint, depending of the velocity (and of developer vacations).

    The team agrees on the Sprint content and commits to it.

    Every developer takes a story. If the story is too big, it is divided in technical tasks and can be worked on in parallel.

    When a story is finished, the product owner validates it.

    \subsubsection{Our work is far to be finished at the end of the Sprint}

    A release candidate is created and intensively tested (on a relatively stable environment).

    Later, this release candidate becomes a release and is deployed in production.

    Many other technical tasks to perform: maintenance, bug fixes, bug hunting in logs (monitoring environments),
    support for other teams,\ldots

    $\Rightarrow$ This is why we do not work on new stories the two last days of a sprint.

    BUILD   -------------- Sprint
    DEVELOP             |   2 days   | for other technical tasks
            -------------------------- Release candidate
            --------------------------------------------------- Release
            --------------------------------------------------------------- Production

    \subsection{Environments}

    \subsubsection{There is a need for multiple environments}

    We need an environment for:

    \begin{itemize}
        \item The developer to code
        \item The developments to be integrated together
        \item The product owner to validate the stories
        \item Testing the release candidate
        \item Pre-production testing (by our commercials and some clients)
        \item The production
    \end{itemize}


    \subsubsection{We need an environment for the developer to code}

    On the developer machine (Linux or Mac)

    Code managed with Git

    IntelliJ as our IDE

    Our microservices architecture makes it challenging to maintain a working
    developer environment

    $\Rightarrow$ Each developer is responsible for his working environment.

    \subsubsection{We need an environment for developments to be integrated together and to validate the stories}

    The BUILD environment is scratched every day.

    The developers can verify if there developments do not break an integrated environment.

    Integration tests run every night.

    Usage of Jenkins for deployment and continuous integration.

    Sadly, this is a very unstable environment.

    $\Rightarrow$ Must be stable at the end of the Sprint.

    \subsubsection{We need an environment for testing the release candidate}


    The STABLEDEV environment. It runs on servers located at Actito headquarters.

    Stabilization of the release candidate.

    Integration tests also run every night.

    One person tests intensively the release candidate.

    This environment is far more stable than Build.

    $\Rightarrow$ Error and instability must be corrected asap.

    \subsubsection{We need an environment for pre-production testing and for production}

    Those environments are called TEST and PROD

    Both used by clients.

    They must obviously be very stable.

    Actito is contractually engaged to guarantee 99\% up-time. (PROD)

    \subsubsection{Each environment is monitored}

    Each environment is under the responsibility of a developer.

    Kibana and Jira are used as bug tracker

    The quality manager and one of the developer are responsible for the bug list.

    Bugs are classified by priority (Blocker, Critical, majori, minor, trivial)

    Not always easy to manage priority between bugs and stories

    Some other monitoring/alerting tools

    KIBANA, Nagios, Munin

    \subsection{Tools}

    \subsubsection{We need tools to support our methodology}

    On a single Git project, developers can work simultaneously on stories, a
    release candidate may exist and the production code needs to stay stable.

    $\Rightarrow$ We use the Git Flow model to address that mess.

    A release (candidate) contains many applications that need

    \subsubsection{Yet Another Deployment Application (YADA)}

    We develop YADA to help us deploy release candidate and release (+- 40 applications)

    \begin{itemize}
        \item List projects that must be released
        \item Deploy and Create automatically RC branch for these projects
        \item Merge into RC branch into develop and master
        \item Deploy project (one by one)
    \end{itemize}

    But it grows, now it's also:

    \begin{itemize}
        \item Check properties and ressources (connections,\ldots)
        \item Puppet Tool (Server configuration --- db, users, tomcat,\ldots)
    \end{itemize}

    puppet

    \subsection{Testing}

    \subsubsection{Testing is a priority}

    A bug in production is absolutely to avoid!

    Unit tests

    In-memory integration tests

    Integration tests

    Product owner manual tests

    Quality manager manual and regression tests

    Put the release candidate ???

    \subsubsection{Unit tests}

    Goal: Detect programming errors quickly and repeatedly

    Advantages:

    \begin{itemize}
        \item Tests the contract of a method
        \item Cheap and quick, good candidate to test limit and error cases
        \item The combination of unit tests validates a functional story
    \end{itemize}

    Drawbacks:
    \begin{itemize}
        \item Hard to write a test independent of other methods
        \item Doesn't catch configuration errors
        \item Hard to test everything (DAOs!)
    \end{itemize}

    \subsubsection{In-memory integration tests}

    Goal: Test an integrated functionality quickly

    Advantages:
    \begin{itemize}
        \item Quick to execute
        \item Detects some configurations errors
        \item Detects some errors on DAOs
    \end{itemize}

    Drawbacks:
    \begin{itemize}
        \item Configuration is fake, not always representative of the reality
        \item Usage of H2 ???
        \item \ldots
    \end{itemize}

    \subsubsection{Integration tests}

    Goal: Test thoroughly an integrated functionality

    Advantages:
    \begin{itemize}
        \item Runs on a real instance of Actito (BUILD, STABLEDEV)
        \item Helps to detect broken functionlities
        \item Runs every night using Jenkins
    \end{itemize}

    Drawbacks:
    \begin{itemize}
        \item Very slow (about 10 hours for Actito Core)
        \item Timeout exceptions
        \item Instability of the BUILD environment
        \item Initial cost of programming the testing environment
    \end{itemize}

    \subsubsection{Manual tests}

    Goal: Find bugs that were not found automatically

    Advantadges:
    \begin{itemize}
        \item It is nearby impossible to write automatic tests for everything
        \item Find interface and browser compatibility bugs
        \item Can detect performance and usability problems
    \end{itemize}

    Drawbacks:
    \begin{itemize}
        \item Requires a lot of man power
        \item Doesn't find everything
        \item Regression tests are hard to perform
    \end{itemize}

    \section{Questions}

    Validation tools? Coding style?
\end{document}
