\documentclass[a4paper, 11pt]{article}

\usepackage[french]{babel}
\usepackage[utf8]{inputenc}
\usepackage[T1]{fontenc}
\usepackage{placeins}
\usepackage{qtree}
\usepackage{csquotes}
\usepackage{hyperref}
\usepackage{amsmath}
\usepackage{tikz}
\usepackage{multicol}
\usetikzlibrary{automata,arrows,positioning,shapes}
\usepackage{graphicx}
\usepackage{amsmath}
\usepackage[left=2cm,right=2cm,top=2cm,bottom=2cm]{geometry}

\graphicspath{{img/}}

\author{Florian Thuin}
\date{\today}
\title{Industrial presentation}

\begin{document}

    \maketitle
    \tableofcontents
    \clearpage{}

    \section{Software development process}

    At Actito, they use the Agile method. The R\&{}D team uses Agile+Scrum and
    the Operations team uses Agile+Kanban. Sprint lasts 2 weeks including a
    2-days test phases. \newline

    Changes during the sprint is made on a Build-branch, at the end of the
    Sprint (once the branch is less unstable), the branch merges on the
    stable-development branch to become a candidate release. Once a candidate
    release is approved by a senior developer, it becomes a release before
    being pushed in production.

    \section{Software development staffing}
    \section{Requirements acquisition process}

    \begin{tikzpicture}[auto, on grid,node distance=3cm, align=center]
        \node[draw, text width=1.5cm] (SM) {Scoping meeting};
        \node[draw, right of=SM, text width=1.5cm] (PO) {Product owner}; % plays the role of a customer
        \node[draw, right of=PO, text width=1.5cm] (LS) {List of stories};
        \node[draw, right of=LS, text width=1.5cm] (Scrum) {Scrum meeting}; % team
        \node[draw, right of=Scrum, text width=1.75cm] (LSPE) {List of stories prioritized evaluates};
        \node[draw, right of=LSPE, text width=1.5cm] (ScrumMaster) {Scrum Master};
        \node[draw, below of=LS, text width=1.85cm] (TeamPP) {Team spokesman};
        \node[draw, below of=ScrumMaster, text width=1.5cm] (SC) {Sprint content};
        \node[above = 2cm of LSPE] (V) {Velocity};
        \node[above = 2cm of ScrumMaster] (A) {Availability};

        \path[-latex] (SM) edge node {} (PO);
        \path[-latex] (PO) edge node {} (LS);
        \path[-latex] (LS) edge node {} (Scrum);
        \path[-latex] (Scrum) edge node {} (LSPE);
        \path[-latex] (LSPE) edge node {} (ScrumMaster);
        \path[-latex] (TeamPP) edge node {} (LS);
        \path[-latex] (ScrumMaster) edge node {} (SC);
        \path[-latex,dashed] (V) edge node {} (ScrumMaster);
        \path[-latex,dashed] (A) edge node {} (ScrumMaster);
    \end{tikzpicture}

    \section{Requirements documentation and specification}

    The product owner is an senior developer from the company that guides and
    specifies the product for the team.

    \section{Software architecture}

    Actito had a monolithic architecture with two big applications: the
    \enquote{Actito Core} and the \enquote{Actito Web}. Those applications were
    hard to maintain and hard to scale. The deployment of a new version generated
    interruption of services and it was a real problem to scale a part of the
    system. If a part of the application required new version of a certain
    technology, the whole application had to be reviewed with the new requirements
    of the new version. \newline

    Since it was no more manageable, the applications were split in small
    single-purpose applications (approx. 40 applications). It improved the
    load-balancing of a part of the system, it decreased the time of interruption
    of service at each release and allowed to perform maintenance only on half
    of the applications last year.

    \section{Quality metrics, measurement}

    There is a team that ensures applications to be user friendly such that
    there is no need of manuals or big documentation (only tooltips inside
    the application). \newline

    No other measurement were discussed.

    \section{Testing process}

    Tests are runned every night and 2 two days at the end of the sprint are
    reserved for manual testing and bug correction.

    \section{Tools used}
    \subsection{Tools used for requirements}

    Requirements are encoded in a tool named Jira. This tool from Atlassian
    create Kanban and Scrum table with the encoded requirements as a to-do list,
    then a developer is assigned to a task, when he starts to work the task moves
    to an in-progress list. When he finished a task, it passes to the code review.
    A manager (senior developer) reviews the code:

    \begin{itemize}
        \item if the work is good, the task passes to a Done list;
        \item if the work isn't good enough, the task goes back to the to-do list
        to be improved.
    \end{itemize}

    \subsection{Tools used for architecture}
    \subsection{Tools used for testing}

    Tests are runned every night on the \enquote{build} environment with Jenkins
    to manage the result of each tests. Kibana, Munin and Nagios are used as bug tracker
    for the bugs not found in the tests (environment are monitored and bugs are
    classified as Blocker, Critical, Major, Minor, Trivial).
    \subsection{Tools used for measurement}

    Measurement of the time needed to complete each requirement is decided by
    planning poker with all developers during a meeting at the start of each
    sprint. \newline

    Measurement of lines of code are made with the Git work-flow, other
    measurements were not discussed.

\end{document}
