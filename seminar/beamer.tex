\documentclass{beamer}

\usepackage{graphicx} % Gestion des images
\usepackage{graphics} % Gestion de placement d'images
\usepackage{hyperref} % Référence liée
\usepackage[english]{babel} % Langue du document
\usepackage[utf8]{inputenc} % Encodage du document
\usepackage[T1]{fontenc} % Encodage de la police
\usepackage[babel=true]{csquotes} % Utilisation d'enquote

\newenvironment{wideitemize}{\itemize\addtolength{\itemsep}{10pt}}{\enditemize}

\mode<presentation>
	{
	\usetheme{contributors}
	%\usetheme{Pittsburgh}
	\setbeamercovered{transparent = 28}
	}

%\usecolortheme{Moo}
%\useinnertheme{Moo}
%\useoutertheme{Moo}


\logo{logo_epl.png}
\title{LINGI2251 Software engineering}
\subtitle{Facts and Fallacies: Management}
\author{Florian Thuin \\ Nicolas Kabasele \\ Nicolas Houtain \\ Nathan Magrofuoco}

\institute{Ecole Polytechnique de Louvain}
\abrevinstitute{Ecole Polytechnique de Louvain}

\begin{document}


\begin{frame}[plain]
	\titlepage
\end{frame}


\AtBeginSection[]
{
   \begin{frame}
       \frametitle{Plan}
       \tableofcontents[currentsection,hideothersubsections]
   \end{frame}
}

\section{Fallacies 3: People}

\subsection{Fact 1}
\begin{frame}
    \frametitle{Fact 1}
    \begin{block}{Fact}
        The most important factor in software work is \textit{not} the tools and
        techniques used by the programmers, but rather the quality of the
        programmers themselves.
    \end{block}
\end{frame}

\subsection{Fact 2}
\begin{frame}
    \frametitle{Fact 2}
    \begin{block}{Fact}
    The best programmers are up to 28 times better than the worst programmers,
    according to \enquote{individual differences} research. Given that
    their pay is never commensurate, they are the biggest bargains in the
    software field.
    \end{block}
\end{frame}

\subsection{Fact 3}
\begin{frame}
    \frametitle{Fact 3}
    \begin{block}{Fact}
    Adding people to a late project makes it later.
    \end{block}
\end{frame}

\subsection{Fact 4}
\begin{frame}
    \frametitle{Fact 4}
    \begin{block}{Fact}
    The working environment has a profound impact on productivity and product
    quality.
    \end{block}
\end{frame}

\section{Fallacies 6: Estimation}

\subsection{Fact 8}
\begin{frame}
    \frametitle{Fact 8}
    \begin{block}{Fact}
    One of the two most common causes of runaway projects is poor estimation.
    \end{block}
\end{frame}

\subsection{Fact 9}
\begin{frame}
    \frametitle{Fact 9}
    \begin{block}{Fact}
    Most software estimates are performed at the beginning of the life cycle.
    This makes sense until we realize that estimates are obtained before the
    requirements are defined and thus before the problem is understood.
    Estimation, therefore, usually occurs at the wrong time.
    \end{block}
\end{frame}

\subsection{Fact 10}
\begin{frame}
    \frametitle{Fact 10}
    \begin{block}{Fact}
    Most software estimates are made either by upper management or by marketing,
    not by the people who will build the software or their managers. Estimation
    is, therefore, done by the wrong people.
    \end{block}
\end{frame}

\subsection{Fact 11}
\begin{frame}
    \frametitle{Fact 11}
    \begin{block}{Fact}
    Software estimates are rarely adjusted as the project proceeds. Thus those
    estimates done at the wrong time by the wrong people are usually not
    corrected.
    \end{block}
\end{frame}

\subsection{Fact 12}
\begin{frame}
    \frametitle{Fact 12}
    \begin{block}{Fact}
    Since estimates are so faulty, there is little reason to be concerned when
    software projects do not meet estimated targets. But everyone is concerned
    anyway.
    \end{block}
\end{frame}

\subsection{Fact 13}
\begin{frame}
    \frametitle{Fact 13}
    \begin{block}{Fact}
    There is a disconnect between management and their programmers. In one
    research study of a project that failed to meet its estimates and was seen
    by its management as a failure, the technical participants saw it as the
    most successful project they had ever worked on.
    \end{block}
\end{frame}

\subsection{Fact 14}
\begin{frame}
    \frametitle{Fact 14}
    \begin{block}{Fact}
    The answer to a feasibility study is almost always \enquote{yes.}
    \end{block}
\end{frame}

\section{Reuse}

\subsection{Fact 15}
\begin{frame}
    \frametitle{Fact 15}
    \begin{block}{Fact}
    Reuse-in-the-small (libraries of subroutines) began nearly 50 years ago and
    is a well-solved problem.
    \end{block}
\end{frame}

\subsection{Fact 16}
\begin{frame}
    \frametitle{Fact 16}
    \begin{block}{Fact}
    Reuse-in-the-large (components) remains a mostly unsolved problem, even
    though everyone agrees it is important and desirable.
    \end{block}
\end{frame}

\subsection{Fact 17}
\begin{frame}
    \frametitle{Fact 17}
    \begin{block}{Fact}
    Reuse-in-the-large works best in families of related systems and thus is
    domain-dependent. This narrows the potential applicability of
    reuse-in-the-large.
    \end{block}
\end{frame}

\subsection{Fact 18}
\begin{frame}
    \frametitle{Fact 18}
    \begin{block}{Fact}
    There are two \enquote{rules of three} in reuse: (a) It is three times as
    difficult to build reusable components as single use components, and (b) a
    reusable component should be tried out in three different applications
    before it will be sufficiently general to accept into a reuse library.
    \end{block}
\end{frame}

\subsection{Fact 19}
\begin{frame}
    \frametitle{Fact 19}
    \begin{block}{Fact}
    Modification of reused code is particularly error-prone. If more than 20 to
    25 percent of a component is to be revised, it is more efficient and
    effective to rewrite it from scratch.
    \end{block}
\end{frame}

\subsection{Fact 20}
\begin{frame}
    \frametitle{Fact 20}
    \begin{block}{Fact}
    Design pattern reuse is one solution to the problems inherent in code reuse.
    \end{block}
\end{frame}

\begin{frame}
	\frametitle{What is this ?}
		\begin{itemize}
 			\item This presentation is based on a simple theme
 			\item This presentation shows how we can customize and make
                \enquote{beamer} ours
 			\item Like the logo on the title slide
 			\item And the logo on the top right corner
 		\end{itemize}
\end{frame}


\begin{frame}
	\frametitle{How tough is it to make a presentation ?}
		\begin{wideitemize}
			\item Not a lot
			\item Simple commands with a concept in mind for
			\pause
			\item Colour
			\pause
			\item Font
			\pause
			\item Inner details
			\pause
			\item Outer details
		\end{wideitemize}
\end{frame}

\section{Colors}

\begin{frame}
	\frametitle{EPL brings colours to life}
		\begin{description}
			\item [Blue] Notice the colour of description item title
			\item [Still blue] It can be set globally
			\item [\color{red} Red] \dots but can also be changed if required
		\end{description}
\end{frame}

\begin{frame}
	\frametitle{You can personalize EPL. (S)he doesn't mind.}

		\begin{theorem}[It's a theorem]
			You can change the way the boxes look and feel
		\end{theorem}

		\begin{exampleblock}{Example}
			Exemplify EPL.
		\end{exampleblock}

		\begin{alertblock}{Alert}
			Sometimes EPL gets carried away.
		\end{alertblock}
\end{frame}


\end{document}
