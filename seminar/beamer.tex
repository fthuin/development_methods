\documentclass{beamer}

\usepackage{graphicx} % Gestion des images
\usepackage{graphics} % Gestion de placement d'images
\usepackage{hyperref} % Référence liée
\usepackage[english]{babel} % Langue du document
\usepackage[utf8]{inputenc} % Encodage du document
\usepackage[T1]{fontenc} % Encodage de la police
\usepackage[babel=true]{csquotes} % Utilisation d'enquote

\newenvironment{wideitemize}{\itemize\addtolength{\itemsep}{10pt}}{\enditemize}

\mode<presentation>
	{
	\usetheme{contributors}
	%\usetheme{Pittsburgh}
	\setbeamercovered{transparent = 28}
	}

%\usecolortheme{Moo}
%\useinnertheme{Moo}
%\useoutertheme{Moo}


\logo{logo_epl.png}
\title{LINGI2251 Software engineering}
\subtitle{Facts and Fallacies: Management}
\author{Florian Thuin \\ Nicolas Kabasele \\ Nicolas Houtain \\ Nathan Magrofuoco}

\institute{Ecole Polytechnique de Louvain}
\abrevinstitute{Ecole Polytechnique de Louvain}

\begin{document}


\begin{frame}[plain]
	\titlepage{}
\end{frame}


\AtBeginSection[]
{
   \begin{frame}
       \frametitle{Plan}
       \tableofcontents[currentsection,hideothersubsections]
   \end{frame}
}

\section{People}

\subsection{Fact 1}
\begin{frame}
    \frametitle{Fact 1}
    \begin{block}{Fact}
        The most important factor in software work is \textit{not} the tools and
        techniques used by the programmers, but rather the quality of the
        programmers themselves.
    \end{block}
    \begin{itemize}
     \pause
     \item \textbf{People matter}.
     \pause
     \item We keep forgetting it for tools, techniques \& processes.
     \pause
     \item Exe : CMM, a good process is the way to good software
     \pause
     \item Fixing people issue is way harder...
     \pause
     \item ... even inventing new technologies seem easier!
    \end{itemize}

\end{frame}

\subsection{Fact 2}
\begin{frame}
    \frametitle{Fact 2}
    \begin{block}{Fact}
    The best programmers are up to 28 times better than the worst programmers,
    according to \enquote{individual differences} research. Given that
    their pay is never commensurate, they are the biggest bargains in the
    software field.
    \end{block}
    \begin{itemize}
     \pause
     \item \textbf{People matter... a lot!}
     \pause
     \item Some programmers are 5 to 28 times better than others.
     \pause
     \item They are the best bargain in software devlopment.
     \pause
     \item But we can't even identify them.
    \end{itemize}

\end{frame}

\subsection{Fact 3}
\begin{frame}
    \frametitle{Fact 3}
    \begin{block}{Fact}
    Adding people to a late project makes it later.
    \end{block}
    \begin{itemize}
     \pause
     \item More than a fact : \textbf{Brook's law} (1995).
     \pause
     \item People added later on a project delay it even more :
     \pause
     \begin{itemize}
      \item They need time to be productive.
      \pause
      \item They need time to learn team communication.
      \pause
      \item \textbf{We} need time to teach them all of these.
     \end{itemize}
     \pause
     \item But not always!
    \end{itemize}

\end{frame}

\subsection{Fact 4}
\begin{frame}
    \frametitle{Fact 4}
    \begin{block}{Fact}
    The working environment has a profound impact on productivity and product
    quality.
    \end{block}
    \begin{itemize}
     \pause
     \item \textbf{Environment matter a lot !}
     \pause
     \item Should facilitate thinking.
     \pause
     \item Distraction and lack of privacy are key enemies to productivity !
     \pause
     \item Counter-example : pair programming.
    \end{itemize}

\end{frame}

\subsection{Fallacy 3}
\begin{frame}
    \frametitle{Fallacy 3}
    \begin{block}{Fallacy}
    Programming can and should be egoless.
    \end{block}
\end{frame}

\section{Estimation}

\subsection{Fact 8}
\begin{frame}
    \frametitle{Fact 8}
    \begin{block}{Fact}
    One of the two most common causes of runaway projects is poor estimation.
    \end{block}
    \pause

    What are runaway projects?
    \begin{itemize}
        \item Projects that are out of control;
        \pause
        \item Projects that are behind schedules;
        \pause
        \item Projects that are over budget.
    \end{itemize}
\end{frame}

\begin{frame}
    \frametitle{Fact 8}
    \begin{block}{Fact}
    One of the two most common causes of runaway projects is poor estimation.
    \end{block}
    \pause

    How \textit{developers} explain the runaway projects?
    \pause
    \begin{itemize}
        \item We use bad tools!
        \pause
        \item We use bad methodologies!
        \pause
        \item We lack of discipline and rigor!
    \end{itemize}
    \bigskip

    \pause
    What are the \textit{real reasons}?
    \begin{itemize}
        \item We did poor estimations (i.e. we were too optimistic);
        \pause
        \item We had unstable requirements
    \end{itemize}
\end{frame}

\begin{frame}
    \frametitle{Fact 8}
    \begin{block}{Fact}
    One of the two most common causes of runaway projects is poor estimation.
    \end{block}
    \pause

    Why are we so bad at estimates?
    \begin{itemize}
        \item We believe in \enquote{expert} people (\enquote{I've been there
        and done that!});
        \pause
        \item We believe in algorithmic approach;
        \pause
        \item We believe in the lines of code approach;
    \end{itemize}
    \bigskip

    \pause
    Are there better approaches?
    \begin{itemize}
        \item The function point (FP) approach: estimates based on the number of
        input and output;
        \pause
        \item The features point approach: estimates based on the number of
        features to be developed;
        \pause
        \item The human-mediated estimation process: mix of the opinion of an
        expert and an algorithm that produces reasonably good answers for this
        family of softwares.
    \end{itemize}
\end{frame}

\subsection{Fact 9}
\begin{frame}
    \frametitle{Fact 9}
    \begin{block}{Fact}
    Most software estimates are performed at the beginning of the life cycle.
    This makes sense until we realize that estimates are obtained before the
    requirements are defined and thus before the problem is understood.
    Estimation, therefore, usually occurs at the wrong time.
    \end{block}
    \pause

    When an software company meets a client, the client wants to know when he
    will get the product. That forces software company to give an estimation
    before starting the work (and so, before starting the requirements
    definition and analysis).
\end{frame}

\subsection{Fact 10}
\begin{frame}
    \frametitle{Fact 10}
    \begin{block}{Fact}
    Most software estimates are made either by upper management or by marketing,
    not by the people who will build the software or their managers. Estimation
    is, therefore, done by the wrong people.
    \end{block}
    \pause

    When estimates are made by the upper management or marketing, they are more
    related to \textit{wishes} than to reality. \newline
    \pause

    One big problem is that there is two points of view:

    \begin{description}
        \item[The political point-of-view]: The contract specifies that the release
        must happen on a certain date.
        \pause
        \item[The rational point-of-view]: It is not possible to develop
        those features with quality requirements at such cost with this schedule.
    \end{description}
\end{frame}

\subsection{Fact 11}
\begin{frame}
    \frametitle{Fact 11}
    \begin{block}{Fact}
    Software estimates are rarely adjusted as the project proceeds. Thus those
    estimates done at the wrong time by the wrong people are usually not
    corrected.
    \end{block}
    \pause

    Even if everyone in the developer team knows the schedule will not be
    respected, nobody will try to create a realistic schedule.
\end{frame}

\subsection{Fact 12}
\begin{frame}
    \frametitle{Fact 12}
    \begin{block}{Fact}
    Since estimates are so faulty, there is little reason to be concerned when
    software projects do not meet estimated targets. But everyone is concerned
    anyway.
    \end{block}
    \pause

    Everyone knows the estimations were bad. But people still give credits to
    them because it's not cool to be late. \newline
    \pause

    Maybe it's our vision of the quality of a project that has to be reviewed.
    Is a good project only a project following its schedule?
\end{frame}

\begin{frame}
    \frametitle{Fact 12}
    \begin{block}{Fact}
    Since estimates are so faulty, there is little reason to be concerned when
    software projects do not meet estimated targets. But everyone is concerned
    anyway.
    \end{block}
    \pause

    What if we managed
    \begin{itemize}
        \item by product (is the product available and working?);
        \pause
        \item by issue (are the issue well and quickly resolved?);
        \pause
        \item by risk (are the identified risks overcome?);
        \pause
        \item by business objectives (did the business performance improve?);
        \pause
        \item by quality (number of quality attributes reached?).
    \end{itemize}
\end{frame}

\begin{frame}
    \frametitle{Fact 12}
    \begin{block}{Fact}
    Since estimates are so faulty, there is little reason to be concerned when
    software projects do not meet estimated targets. But everyone is concerned
    anyway.
    \end{block}

    eXtreme Programming is another approach, the customer can choose 3 out of the
    4 factors:
    \begin{itemize}
        \item cost;
        \pause
        \item schedule;
        \pause
        \item features;
        \pause
        \item quality.
    \end{itemize}
\end{frame}

\subsection{Fact 13}
\begin{frame}
    \frametitle{Fact 13}
    \begin{block}{Fact}
    There is a disconnect between management and their programmers. In one
    research study of a project that failed to meet its estimates and was seen
    by its management as a failure, the technical participants saw it as the
    most successful project they had ever worked on.
    \end{block}
    Why management consider it as a failure?
    \begin{itemize}
    	\item 27 months instead of 14.
    	\item 130\% for the sotfware.
    	\item 800\% for the firmware.
    \end{itemize}
\end{frame}

\begin{frame}
	\frametitle{Fact 13}
	Why programmer consider it a success?
	\begin{itemize}
		\item Final product was complete.
		\item Developing was a technical challenge.
		\item Freedom given by the management.
	\end{itemize}
	The main problem is to define what's constitute a succesful project.
\end{frame}

\subsection{Fact 14}
\begin{frame}
    \frametitle{Fact 14}
    \begin{block}{Fact}
    The answer to a feasibility study is almost always \enquote{yes.}
    \end{block}
    \begin{itemize}
    		\item Developer tends to be too optimistic.
    		\item Illusion that everything will go without problem.
    		\item Time frame between the answer and discovery of its falseness too long.
    \end{itemize}
\end{frame}

\subsection{Fallacy 6}
\begin{frame}
    \frametitle{Fallacy 6}
    \begin{block}{Fallacy}
    To estimate cost and schedule, first estimate lines of code.
    \end{block}
    \begin{itemize}
    		\item Conversion from LOC to cost and schedule
    		\item Fallacy?
    		\begin{itemize}
    			\item No reason more reliable cost/schedule estimation.
    			\item No universal conversion technique (programming language)	
    		\end{itemize}
    \end{itemize}
\end{frame}

\section{Reuse}

\subsection{Fact 15}
\begin{frame}
    \frametitle{Fact 15}
    \begin{block}{Fact}
    Reuse-in-the-small (libraries of subroutines) began nearly 50 years ago and
    is a well-solved problem.
    \end{block}
    \begin{itemize}
    		\item First libraries of subroutines in the mid-1950s.
    		\item Math functions, String handlers,sorts,\ldots
    		\item Succesful idea which try to be expanded.
    \end{itemize}
\end{frame}

\subsection{Fact 16}
\begin{frame}
    \frametitle{Fact 16}
    \begin{block}{Fact}
    Reuse-in-the-large (components) remains a mostly unsolved problem, even
    though everyone agrees it is important and desirable.
    \end{block}
    \begin{itemize}
    		\item More difficult to build a large reusable component than a small one.
    		\item Two points of view:
    		\begin{itemize}
    			\item Futur of the field, where program are an aggregate of components.
    			\item Nearly possible to generalize enough.
    		\end{itemize}
    \end{itemize}
\end{frame}

\begin{frame}
	\frametitle{Fact 16}
	\begin{itemize}
		\item It depend on sofware diversity, is there enough common problem across project?
		\item \textit{Not-Invented-Here} phenomenon
		\item A lack of skill?
	\end{itemize}

\end{frame}

\subsection{Fact 17}
\begin{frame}
    \frametitle{Fact 17}
    \begin{block}{Fact}
    Reuse-in-the-large works best in families of related systems and thus is
    domain-dependent. This narrows the potential applicability of
    reuse-in-the-large.
    \end{block}

    \begin{itemize}

    \item Reuse-in-the-large is \textbf{difficult}
    \begin{itemize}
        \item[\alert{because}] need component connection across
            application domain
        \item[\alert{but}] within a specific domain it's easy
    \end{itemize}

    \item \textit{One-size-fits-all} tools is always a believe of latter
    people

        $\Rightarrow$ They are wrong to thinks that construction of
        software are the same for all domain

        \end{itemize}

\end{frame}

\subsection{Fact 18}
\begin{frame}
    \frametitle{Fact 18}
    \begin{block}{Fact}
    There are two \enquote{rules of three} in reuse: (a) It is three times as
    difficult to build reusable components as single use components, and (b) a
    reusable component should be tried out in three different applications
    before it will be sufficiently general to accept into a reuse library.
    \end{block}

    \begin{itemize}
        \item[$\Rightarrow$] Couple of rules of thumb

        \item[(a)] Reusable component (RC) need:
            \begin{enumerate}
                \item Find the "general problem"
                \item Build the RC such at the general and specific is
                    solved by the same way
                \item Find a "general testing approach"
            \end{enumerate}

            $\rightarrow$  3 come from knowledgeable reuse expert

        \item[(b)] arbitrary (magical?) number which is more "at
            least" three times $\Rightarrow$ implies minimal reusability
    \end{itemize}
\end{frame}



\begin{frame}
    \frametitle{Fact 18}
    \begin{block}{Fact}
    There are two \enquote{rules of three} in reuse: (a) It is three times as
    difficult to build reusable components as single use components, and (b) a
    reusable component should be tried out in three different applications
    before it will be sufficiently general to accept into a reuse library.
    \end{block}


    \begin{itemize}

        \item Everyone acknowledge that RC take more time: more think,
            more verification.

        \item Number three is accepted:
            \begin{itemize}
                \item[\alert{because}] it's a rules of thumb
            \item[\alert{anyone}] has the conviction that it's the exact number and
            that it MUST NEVER CHANGE
            \end{itemize}
    \end{itemize}
\end{frame}

\subsection{Fact 19}
\begin{frame}
    \frametitle{Fact 19}
    \begin{block}{Fact}
    Modification of reused code is particularly error-prone. If more than 20 to
    25 percent of a component is to be revised, it is more efficient and
    effective to rewrite it from scratch.
    \end{block}

    Hard to have reuse-in-the-large: why not modify component?
    %to fit the problem at hand?

    \begin{enumerate}
        \item Software design = framework + philosophy

            $\rightarrow$ modification = understand
            framework and accepts philosophy

        \item Design envelop constraint problem that not fit in the
            envelope

        \item Need to "comprehending the existing solution"

            $\rightarrow$ programmer who originally built the solution
            may find it difficult to modify some months later.

        \item[$\Rightarrow$] Documentation is the solution to these
            problem but:
            \begin{itemize}
                    \item Documentation is for almost NIL
                    \item If there is documentation he is not update on
                        modification
            \end{itemize}

            $\rightarrow$ No maintenance documentation at the end
    \end{enumerate}

\end{frame}

\begin{frame}
    \frametitle{Fact 19}
    \begin{block}{Fact}
    Modification of reused code is particularly error-prone. If more than 20 to
    25 percent of a component is to be revised, it is more efficient and
    effective to rewrite it from scratch.
    \end{block}

    Easy to accept if accept \textit{"software products are difficult to
    build and maintain"}
    \begin{itemize}
        \item[$\Rightarrow$] People who don't accept this have
            \begin{itemize}
                \item Never see complicated software solution
                \item As play with toy problems only
                \item Only exposed to software through literacy
                    course with complicated software such as
                    "Hello world"
            \end{itemize}

            For those people, let them do shit.
    \end{itemize}

\end{frame}


\begin{frame}
    \frametitle{Fact 19}
    \begin{block}{Fact}
    Modification of reused code is particularly error-prone. If more than 20 to
    25 percent of a component is to be revised, it is more efficient and
    effective to rewrite it from scratch.
    \end{block}

    Corollary with \textit{"It is almost always a mistake to modify packaged,
    vendor-produced software systems."}

    \begin{itemize}
        \item Many different release (added functionality, solve problem)

            \begin{itemize}
                \item[\alert{IF}] Approach review \alert{THEN} old modification remake
                \item[\alert{SO}] Financial cost + moral cost
            \end{itemize}

    \end{itemize}
\end{frame}


\subsection{Fact 20}
\begin{frame}
    \frametitle{Fact 20}
    \begin{block}{Fact}
    Design pattern reuse is one solution to the problems inherent in code reuse.
    \end{block}

    Reusing code Vs Reusing design = \textbf{design pattern}

    \begin{itemize}
        \item It's a description of a problem and a solution
            (When apply and the consequences)
        \item More conceptual and abstract than code itself

        \item[Note]: emerge from practice not theory
    \end{itemize}

    Adopted by practitioners AND academics but difficult to measure the
    impact of this new work on practice.

    $\Rightarrow$  Design patterns represent one of the most unequivocally satisfying,
    least forgotten, truths of the software field.

\end{frame}

\section{Questions}

\begin{frame}[plain,c]
%\frametitle{A first slide}

\begin{center}
\Huge Questions ?
\end{center}

\end{frame}

\end{document}
